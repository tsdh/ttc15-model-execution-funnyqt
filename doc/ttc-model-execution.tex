\documentclass[submission]{eptcs}
\providecommand{\event}{TTC 2015}

\usepackage[T1]{fontenc}
\usepackage{varioref}
\usepackage{hyperref}

\usepackage{url}
\usepackage{paralist}
\usepackage{graphicx}
\usepackage[cache]{minted}
\newminted{clojure}{fontsize=\fontsize{8}{8},linenos,numbersep=3pt,numberblanklines=false}
\newmintinline{clojure}{fontsize=\small}
\newcommand{\code}{\clojureinline}

\title{Solving the TTC Model Execution Case with FunnyQT}
\author{Tassilo Horn
  \institute{Institute for Software Technology, University Koblenz-Landau, Germany}
  \email{horn@uni-koblenz.de}}

\def\titlerunning{Solving the TTC Model Execution Case with FunnyQT}
\def\authorrunning{T. Horn}

\begin{document}
\maketitle

\begin{abstract}
  This paper describes the FunnyQT solution to the TTC 2015 Model Execution
  transformation case.  The solution solves the third variant of the case,
  i.e., it considers and implements the execution semantics of the complete UML
  Activity Diagram language.

  FunnyQT is a model querying and model transformation library for the
  functional Lisp-dialect Clojure providing a comprehensive and efficient
  querying and transformation API, many parts of which are provided as
  task-oriented embedded DSLs.
\end{abstract}


\section{Introduction}
\label{sec:introduction}

This paper describes the FunnyQT\footnote{\url{http://funnyqt.org}}
~\cite{Horn2013MQWFQ} solution of the TTC 2015 Model Execution
Case~\cite{model-execution-case-desc}.  It implements the third variant of the
case description, i.e., it implements the execution semantics of the complete
UML Activity Diagram language.  The solution project is available on
Github\footnote{\url{https://github.com/tsdh/ttc15-model-execution-funnyqt}},
and it is set up for easy reproduction on a SHARE image\footnote{\url{FIXME:
    add SHARE URL}}.

FunnyQT is a model querying and transformation library for the functional Lisp
dialect Clojure\footnote{\url{http://clojure.org}}.  Queries and
transformations are plain Clojure programs using the features provided by the
FunnyQT API.

As a Lisp, Clojure provides strong metaprogramming capabilities that are
exploited by FunnyQT in order to define several \emph{embedded domain-specific
  languages} (DSL, \cite{book:Fowler2010DSL}) for different querying and
transformation tasks.

FunnyQT is designed with extensibility in mind.  By default, it supports EMF
\cite{Steinberg2008EEM} models and
JGraLab\footnote{\url{http://jgralab.github.io}} TGraph models.  Support for
other modeling frameworks can be added without having to touch FunnyQT's
internals.

The FunnyQT API is structured into the following namespaces, each namespace
providing constructs supporting concrete querying and transformation use-cases:

\begin{compactdesc}
\item[funnyqt.emf] EMF-specific model management API
\item[funnyqt.tg] JGraLab/TGraph-specific model management API
\item[funnyqt.generic] Protocol-based, generic model management API
\item[funnyqt.query] Generic querying constructs such as quantified expressions
  or regular path expressions
\item[funnyqt.polyfns] Constructs for defining polymorphic functions
  dispatching on metamodel types
\item[funnyqt.pmatch] Pattern matching constructs
\item[funnyqt.relational] Constructs for logic-based, relational model querying
  inspired by Prolog
\item[funnyqt.in-place] In-place transformation rule definition constructs
\item[funnyqt.model2model] Out-place transformation definition constructs
  similar to ATL or QVT Operational Mappings
\item[funnyqt.extensional] Transformation API similar to GReTL
\item[funnyqt.bidi] Constructs for defining bidirectional transformations
  similar to QVT Relations
\item[funnyqt.coevo] Constructs for transformations that evolve a metamodel and
  a conforming model simultaneously at runtime
\item[funnyqt.visualization] Model visualization
\item[funnyqt.xmltg] Constructs for querying and modifying XML files as models
  conforming to a DOM-like metamodel
\end{compactdesc}

For solving the model execution case, the \emph{funnyqt.emf},
\emph{funnyqt.query}, and \emph{funnyqt.polyfns} namespaces have been used.


\section{Solution Description}
\label{sec:solution-description}

The explanations in the case description about the operational semantics on UML
Activity Diagrams suggest an algorithmic solution to the transformation case.
The FunnyQT solution tries to be almost a literal translation of the case
description to Clojure code.

The first line of the solution calls the \code|generate-ecore-model-functions|
FunnyQT macro.

\begin{clojurecode}
(generate-ecore-model-functions "activitydiagram.ecore" ttc15-model-execution-funnyqt.ad a)
\end{clojurecode}

As its name suggests, it generates a metamodel-specific API.  This API is
generated into the namespace \code|ttc15-model-execution-funnyqt.ad|, and the
namespace alias \code|a| is used to refer to that namespace from the current
one.

The generated API consists of element creation functions, lazy element sequence
functions, attribute access functions, and reference access functions.  For
example, \code|(a/create-ControlToken! ad)| creates a new control token and
adds it to the activity diagram model \code|ad|, \code|(a/eall-Inputs ad)|
returns the lazy sequence of input elements in \code|ad|, \code|(a/running? a)|
and \code|(a/set-running! a true)| query and set the action \code|a|'s
\textsf{running} attribute, and \code|(a/->locals a)|, \code|(a/->set-locals! a
ls)|, and \code|(a/->add-locals! a l)| query, set, and add to the
\textsf{locals} reference of the activity \code|a|.





\section{Evaluation}
\label{sec:evaluation}



\section{Conclusion}
\label{sec:conclusion}



\bibliographystyle{eptcs}
\bibliography{ttc-model-execution}
\end{document}

%%% Local Variables:
%%% mode: latex
%%% TeX-master: t
%%% TeX-command-extra-options: "-shell-escape"
%%% End:

%  LocalWords:  parallelizes
